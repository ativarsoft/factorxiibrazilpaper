\documentclass{article}
\begin{document}
\title{Produtos alimentícios vendidos no Brasil para pessoas com uma mutação do Factor XII}
\author{Ai Nakamoto}
\date{August 2023}
\maketitle

\begin{abstract}
Indivíduos com mutação de fator XII devem evitar
legumes, nozes, castanhas, amêndoas e tubérculos.
\end{abstract}

\section{Introdução}

O uso da mídia como ferramenta política de controle social
tem prejudicado o desenvolvimento de tecnologias relacionadas
a bioinformática e engenharia genética. Muitas das vezes
a análise de material genético de um indivíduo é vista com
maus olhos devido à política anti-racista vigente.
A análise de material genético de indivíduos pode ser útil
para identificar condições de saúde de cunho genético que
podem condicionar a vida e o bem-estar de um indivíduo e
da sociedade consequentemente.

Este estudo procura identificar alimentos vendidos atualmente
no Brasil que podem causar reações alérgicas em indivíduos com
uma mutação de fator XII.

\section{Metodologia}

Indivíduos afetados por mutação de fator XII foram submetidos a uma dieta
regular e rígida com base em pães, macarrão, vegetais folhosos, frutas sem
as sementes, ovos e carne. Sal e glutamato
monosódico, alho e cebola foram utilizados como tempero. Todos os indivíduos
se recuperaram.

Cinco casos de comorbidade de mutação de fator XII e síndrome de William
foram observados. Todos os indívidos fizeram a dieta sugerida e vivem sem
nenhum sintoma típicos de inflamação causada pelo sistema imunológico.
Nunhum deles precisou de qualquer medicação psiquiátrica após iniciar a
dieta sugerida.

\section{Principais alérgenos}

Os principais alérgenos a indivíduos com mutação em fator XII são tubérculos, beringela; pimentas; pimentão verde; e tomates.

Cafeína, tabaco e cacau, ambrósia apresentam reação cruzada com o Fator XII.

Alguns alimentos e temperos que não são comuns na culinária internacional e
que estão presentes com frequência na culinária brasileira causam reações
alérgicas em indivíduos com a mutação. A mandioca é um tubérculo e urucum
é uma semente usada como colorífico. Reações alérgicas ocorreram em decorrência
à ingestão desses alimentos em indivíduos afetados.

Maionese Hellmann's contém páprica.
Muitos macarrões comercializados em supermercado contém cúrcuma e urucum.
Macarrões sem glútem menos frequentemente contém esse tipo de tempero.
Muitos temperos para carne e condimentos preparados vendidos em supermercado
contém alérgenos.

\section{Nozes e castanhas}

Certas nozes e sementes cruas, como amêndoas, castanha de caju, sementes de linhaça e sementes de gergelim, contêm inibidores de tripsina. Assar ou cozinhar essas nozes e sementes pode ajudar a reduzir os níveis de inibidores de tripsina.

Refrigerantes de cola contém extrato de noz de cola.
Refrigerantes de guaraná contém sementes de guaraná.

Maionese Heiz contém farinha de mostarda.

\section{Tubérculos}

Batatas cruas contêm inibidores de tripsina, particularmente na forma de glicoalcalóides, que são compostos naturais encontrados nas batatas. No entanto, cozinhar batatas em temperaturas acima de 80°C (176°F) pode reduzir significativamente os níveis de inibidores de tripsina.

Mandioca, batata-doce, cenoura, nabo e beterrabo são alguns exemplos de
tubérculos que causaram reações alérgicas.

\section{Legumes}

Legumes como soja, feijões, e feijão-de-lima contêm inibidores de tripsina.
Esses inibidores estão presentes em legumes crus ou mal cozidos.
No entanto, os processos de cozimento, imersão ou fermentação podem ajudar a reduzir os níveis de inibidores de tripsina nas leguminosas.

\section{Milho}

O inibidor de tripsina de milho é conhecido por causar reações
alérgicas em indivíduos com uma mutação de fator XII.

\section{Considerações psiquiátricas}

Indivíduos com uma mutação do fator XII parecem ter uma predisposição
a inteligência superior. Indivídos com alterações genéticas que causam
síndrome de William também tem uma predisposição a inteligência superior.
Inteligência muito acima da média foi observada em indivídos com comorbidade
dessas duas alterações.

Tratamentos medicamentosa psiquiática não é normalmente necessários
em indivídos
com mutação de fator XII que seguem uma dieta rigorosa que não tem a
presença dos alérgenos.

Vários casos de indivíduos afetados que não receberam tratamento adequado
e foram erroneamente e forçosamente internados em hospitais psiquiátrico
desnecessáriamente pelo estado ou pela família do paciente por
desconhenhecimento e despreparo de médicos brasileiros foram observados.
Vários casos de indivídos que fugiram do país tembém foram observados.

\section{Considerações patofisiológicas}

Nas instâncias em que há ingestão acidental um ou mais alérgenos, mesmo
que em pequenas quantidades, indivíduos afetados apresentam alteração de
humor que dura até a eliminação dos alérgenos do corpo por meios naturais.
Esses indivíduos podem se apresentar nervosos, irritados ou estressados
nesses casos de ingestão acidental. Pode haver diminuição da pressão
sanguínea, fechamento to trato digestório superior e outros sintomas que
são comuns para os casos de outras alergias.

Exposição prolongada a alérgenos causa stress oxidativo. Os cabelos
apresentam fios brancos em indivídos jovens. Dificuldade de ereção pode
ocorrer durante as crises causadas por ingestão acidental em indivíduos
jovens do sexo masculino.

Indivídos afetados que seguem uma dieta rigorosa não apresentam nenhum
sintoma de alteração fisiológica.

\section{Referências}

Hamad BK, Pathak M, Manna R, Fischer PM, Emsley J, Dekker LV. Assessment of the protein interaction between coagulation factor XII and corn trypsin inhibitor by molecular docking and biochemical validation. J Thromb Haemost. 2017 Sep;15(9):1818-1828. doi: 10.1111/jth.13773. Epub 2017 Aug 9. PMID: 28688220; PMCID: PMC5638086.

\end{document}

