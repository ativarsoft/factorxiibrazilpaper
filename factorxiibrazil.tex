\documentclass{article}
\begin{document}
\title{Produtos alimentícios vendidos no Brasil para pessoas com uma mutação do Factor XII}
\author{Ai Nakamoto}
\date{August 2023}
\maketitle

\begin{abstract}
Indivíduos com mutação de fator XII devem evitar
legumes, nozes, castanhas, amêndoas e batatas.
\end{abstract}

\section{Introdução}

O uso da mídia como ferramenta política de controle social
tem prejudicado o desenvolvimento de tecnologias relacionadas
a bioinformática e engenharia genética. Muitas das vezes
a análise de material genético de um indivíduo é vista com
maus olhos devido à política anti-racista vigente.
A análise de material genético de indivíduos pode ser útil
para identificar condições de saúde de cunho genético que
podem condicionar a vida e o bem-estar de um indivíduo e
da sociedade consequentemente.

Este estudo procura identificar alimentos vendidos atualmente
no Brasil que podem causar reações alérgicas em indivíduos com
uma mutação de fator XII.

\section{Nozes e castanhas}

Certas nozes e sementes cruas, como amêndoas, castanha de caju, sementes de linhaça e sementes de gergelim, contêm inibidores de tripsina. Assar ou cozinhar essas nozes e sementes pode ajudar a reduzir os níveis de inibidores de tripsina.

Refrigerantes de cola contém extrato de noz de cola.
Refrigerantes de guaraná contém sementes de guaraná.

Maionese Heiz contém farinha de mostarda.
Não foram identificados alérgenos em maionese Hellmann's.

\section{Legumes}

Legumes como soja, feijões, e feijão-de-lima contêm inibidores de tripsina.
Esses inibidores estão presentes em legumes crus ou mal cozidos.
No entanto, os processos de cozimento, imersão ou fermentação podem ajudar a reduzir os níveis de inibidores de tripsina nas leguminosas.

Batatas cruas contêm inibidores de tripsina, particularmente na forma de glicoalcalóides, que são compostos naturais encontrados nas batatas. No entanto, cozinhar batatas em temperaturas acima de 80°C (176°F) pode reduzir significativamente os níveis de inibidores de tripsina.

\section{Milho}

O inibidor de tripsina de milho é conhecido por causar reações
alérgicas em indivíduos com uma mutação de fator XII.

\section{Considerações psiquiátricas}

Indivíduos com uma mutação do fator XII parecem ter uma predisposição
a inteligência superior.

\section{Referências}

Hamad BK, Pathak M, Manna R, Fischer PM, Emsley J, Dekker LV. Assessment of the protein interaction between coagulation factor XII and corn trypsin inhibitor by molecular docking and biochemical validation. J Thromb Haemost. 2017 Sep;15(9):1818-1828. doi: 10.1111/jth.13773. Epub 2017 Aug 9. PMID: 28688220; PMCID: PMC5638086.

\end{document}

